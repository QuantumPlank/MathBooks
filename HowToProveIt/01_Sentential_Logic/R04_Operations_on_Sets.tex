\textbf{Definition} \textit{The intersection of two sets $A$ and $B$ is the set $A \cap B$ defined as} \[A \cap B = \{x \mid x \in A \text{ and }x \in B\}\]

\textbf{Definition} \textit{The union of two sets $A$ and $B$ is the set $A \cup B$ defined as} \[A \cup B = \{x \mid x \in A \text{ or } x \in B\}\]

\textbf{Definition} \textit{The difference of $A$ and $B$ is the set $A \setminus B$ defines as} \[A \setminus B = \{x \mid x \in A \text{ and } x \notin B\}\]

Sometimes it is helpful when working with operations on sets to draw pictures of the results of these operations, the most common diagram for this is a \textit{Venn Diagram}.

\textbf{Definition} \textit{The symmetric difference of $A$ and $B$ is the set $A \triangle B$ defines as} \[A \triangle B = \{x \mid x \in A \text{ and } x \in B \text{ and } x \notin A \cap B\}\] \[A \triangle B = (A \setminus B) \cup (B \setminus A) = (A \cup B) \setminus (A \cap B)\]

Set theory operations are related to the logical connectives, but it is important to remember that although they are related, they are not interchangeable. The logical connectives can only be used to combine statements, whereas the set theory operations must be used to combine sets.

\textbf{Definition} \textit{Supossing that $A$ and $B$ are sets. We will say that $A$ is a subset of $B$ if every element of $A$ is also an element of $B$. We write $A \subseteq B$ to mean that $A$ is a subset of $B$. $A$ and $B$ are said to be disjoint if they have no elements in common, that is if $A \cap B = \emptyset$}

\textbf{Theorem} \textit{For any sets $A$ and $B$, $(A \cup B) \setminus B \subseteq A$}