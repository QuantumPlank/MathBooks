It is often necesary to make statements about objects that are represented by letters called \textit{variables}.

A notation like \(P(x)\) can be interpreted as a statement \(P\) about the variable \(x\).

In a statement that contains variables we cannot describe the statement as being simply true or false. Its truth value might depend on the values of the variables involved.

A \textit{set} is a collection of objects. The objects in the collection are called the \textit{elements} of the set.

We use the symbol \(\in\) to mean \textit{is an element of}. To say that an element is not part of a set we use the symbol \(\notin\).

The following notations for sets are valid:
\begin{itemize}
    \item \(A = \{1,2,3\}\) is set that contains the numbers 1, 2 and 3
    \item \(B = \{2,3,5,7,11,13,17,...\}\) is a set that contains all of the prime numbers
    \item \(C = \{x \mid x \text{ is a prime number}\}\) is a set that defines an \textit{elementhood test} for the set, any value of x that passes the test is an element of the set.
\end{itemize}

\textit{Free} variables in a statement stand for objects that the statement says something about. \textit{Bound} variables are simply letters that are used as a convenience to help express an idea and should not be thought of as standing for any particular object. A bound variable can always be replaced by a new variable without chaning the meaning of the statement, and often the statement can be rephrased so that the bound variables are eliminated altogether.

To distinguish the values of a statement that contains free variables that make the statement true from those that make it false, we form the set of values of the free variables for which the statement is true. We call this the \textit{truth set} of the statement.

For a statement, the set of all the objects that represent the free variables is called the \textit{universe of discourse} for the statement. And we say that the variables \textit{range over} this universe.

A set without any elements is called an \textit{empty set}, or the \textit{null set}, and it is often denoted by $\emptyset$