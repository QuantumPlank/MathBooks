High school mathematics is concerd mostly with solving equations and computing answers to numerical questions. College mathematics deals with a wider variety of questions, involving not only numbers, but also sets, functions, and other mathematical objects. What ties them together is the use of deductive reasoning to find the answers to questions.

\textit{Deductive reasoning} uses general ideas to come to a specific conclusion. Deductive reasoning in mathematics is usually presented in the form of a \textit{proof}.

A number is \textit{prime} of it cannot be written as a product of two smaller positive integers. If it can be written as a product of two smaller positive integers, then it is \textit{composite}

A \textit{conjecture} is a conclusion that is proffered on a tentative basis without proof.

\textbf{Conjecture 1.} \textit{Suppose \(n\) is an integer larger than \(1\) and \(n\) is prime. Then \(2^n - 1\) is prime}.

\textbf{Conjecture 2.} \textit{Suppose \(n\) is an integer larger than \(1\) and \(n\) is not prime. Then \(2^n - 1\) is not prime}.

A \textit{counterexample} is a specific instance that demonstrates the falsity of a general statement, argument or theory.

The existence of even one counterexample establishes that a conjecture is incorrect. However, failure to find a counterexample to a conjecture does not show that the conjecture is correct.

We can never be sure that the conjecture is correct if we only check examples. No matter how many examples we check, there is always the possibility that the next one will be the first counterexample.

Once a conjecture has been proven, we can call it a \textit{theorem}.

\textbf{Theorem 3.} \textit{Suppose \(n\) is an integer larger than \(1\) and \(n\) is not prime. Then \(2^n - 1\) is not prime}

Prime numbers of the form \(2^n -1\) are called \textit{Mersenne primes}. Although many Mersenne primes have been found, it is still not know if there are infinitely many of them.

A positive integer \(n\) is said to be \textit{perfect} if \(n\) is equal to the sum of all positive integers smaller than \(n\) that divide  \(n\).

For any two integers \(m\) and \(n\), we say that \(m\) \textit{divides} \(n\) if \(n\) is divisible by \(m\); in other words, if there is an integer \(q\) such that \(n=qm\).

For any positive integer \(n\), the product of all integers from \(1\) to \(n\) is called \(n\) \textit{factorial} and is denoted \(n!\).

\textbf{Theorem 4.} \textit{For every positive integer \(n\), there is a sequence of \(n\) consecutive positive integers containing no primes}

Pairs of primes that differ by only two are called \textit{twin primes}
