\begin{enumerate}
    %1
    \item
    ($\rightarrow$) Supposing that $\forall x (P(x) \land Q(x))$, let $y$ be arbitrary. Then $P(y) \land Q(y)$, since $P(y)$ for an arbitrary $y$, $\forall x P(x)$, similarly since $Q(y)$ for an arbitrary $y$, $forall y Q(y)$. Thus, $\forall x P(x) \land \forall x Q(x)$
    ($\leftarrow$) Supposing that $\forall x P(x) \land \forall x Q(x)$, let $y$ be arbitrary. Then since $\forall x P(x)$, $P(y)$, similarly for $forall x Q(x)$, $Q(x)$, Thus $P(y) \land P(y)$ and since $y$ was arbitrary, it follows that $\forall x (P(x) \land Q(x))$
    %2
    \item
    Supposing that $A \subseteq B$ and $A \subseteq C$ are true. Let $x$ be an arbitrary element of $A$. Since $A \subseteq B$, $x \in B$ and similarly since $A \subseteq C$, $x \in C$. Thus $x \in B \cap C$, but since $x$ was an arbitrary element of $A$, then $A \subseteq B \cap C$
    %3
    \item
    Supposing that $A \subseteq B$. Let $C$ be an arbitrary set. Let $x$ be an arbitrary element of $C \setminus B$. Since $x \in C \setminus B$, then $x \in C$ and $x \notin B$. If $x \in A$, then since $A \subseteq B$, $x \in B$ but $x \notin B$, thus $x \notin A$. Since $x \in C$ and $x \notin A$, $x \in C \setminus A$. But $x$ was an arbitrary element of $C \setminus B$. Thus $C \setminus B \subseteq C \setminus A$
    %4
    \item
    Supposing that $A \subseteq B$ and that $A \not \subseteq C$. Let $x$ be an arbitrary element of $A$, then since $A \not \subseteq C$, $x \notin C$, and since $A \subseteq B$, $a \in B$. Thus since $x \in B$ but $x \notin C$, $B \not \subseteq C$
    %5
    \item
    Suppose that $A \subseteq B \setminus C$ and that $A \neq \emptyset$. Since $A \neq \emptyset$, let $x$ be an arbitrary element of $A$. Since $A \subseteq B \setminus C$, $x \in B \setminus C$, which means that $x \in B$ and $x \notin C$. Thus since $x \in B$ but $x \notin C$, $B \not \subseteq C$
    %6
    \item
    Let $x$ be an arbitrary element of $A \setminus (B \cap C)$ \\
    $x \in A \setminus (B \cap C) \text{ iff } x \in A \land x \notin B \cap C$ \\
    $\text{iff } x \in A \land \lnot (x \in B \land x \in C)$\\
    $\text{iff } x \in A \land (x \notin B \lor x \notin C)$\\
    $\text{iff } (x \in A \land x \notin B) \lor (x \in A \land x \notin C)$\\
    $\text{iff } x \in (A \setminus B) \cup (A \setminus C)$
    %7
    \item
    Suppose that $\mathscr{P}(A \cap B)$ Let $X$ be an arbitrary set such that $X \in \mathscr{P}(A \cap B)$ and let $x$ be an arbitrary element of $X$. Since $X \in \mathscr{P}(A \cap B)$, $X \subseteq (A \cap B)$, and since $x \in X$, $x \in A \land x \in B$. Since $x$ was arbitrary this means that $X \subseteq A$ and $X \subseteq B$, which in turn means that $X \in \mathscr{P}(A)$ and $X \in \mathscr{P}(B)$.
    Now supposing that $\mathscr{P}(A) \cap \mathscr{B}$. Let $X$ be an arbitrary element of $\mathscr{P}(A) \cap \mathscr{B}$. This means that $X \in \mathscr{P}(A) \land X \in \mathscr{P}(B)$, let $x$ be an arbitrary element of $X$. Then $X \subseteq A$, which means that $x \in A$, similarly since $X \in \mathscr{P}(B)$, $x \in B$, thus $x \in A \cap B$, which means that $X \in \mathscr{P}(A \cap B)$
    %8
    \item
    Supposing that $A \subseteq B$. Let $X$ be an arbitrary element of $\mathscr{P}(A)$, then $X \subseteq A$. Let $x$ be an arbitrary element of $X$, such that $x \in A$. Since $A \subseteq B$, $x \in B$. Since $x$ is an arbitrary element of $X$, it follows that $X \in \mathscr{P}(B)$. Thus $\mathscr{P}(A) \subseteq \mathscr{B}$.
    Now suppose that $\mathscr{P}(A) \subseteq \mathscr{P}(B)$. Let $X$ be an arbitrary element of $\mathscr{P}(A)$ and let $x$ be an arbitrary element of $X$, then $x \in A$, suppose that $A \not \subseteq B$, thus $x \not in B$. Since $\mathscr{P}(A) \subseteq \mathscr{P}(B)$, $X \in \mathscr{P}(B)$, which means that $x \in B$, but $x \notin B$, which is a contradiction, thus $A \subseteq B$
    %9
    \item
    Suppose that $x$ and $y$ are odd integers. This means that for some integer $k$, $x = 2k + 1$, and that for some integer $m$, $y = 2m + 1$. Multiplying $x$ with $y$ we obtain that $xy = (2k + 1)(2m + 1) = (4km + 2k + 2m + 1)$. Simplifying we see that $(4km + 2k + 2m + 1) = 2(2km + k + m) + 1$, since $2km + k + m$ is an integer $xy$ is odd.
    %10
    \item
    Suppose that $x$ and $y$ are odd integers. Then for some integers $k$ and $m$, $x = 2k + 1$ and $y = 2m + 1$. Subtracting $y$ from $x$, we see that $x - y = (2k + 1) - (2m + 1) = 2k -2m + 1 - 1 = 2k - 2m = 2(k - m)$. Since $k - m$ is an integer, by the definition of even number $x - y$ is even.
    %11
    \item
    Let $n$ be an arbitrary integer. Suppose that $n^3$ is even. Suppose that $n$ is odd, then for some integer $k$, $n = 2k + 1$. Then raising $n$ to the 3rd power, $n^3 = (2k + 1)^3 = 8k^3 + 4k^2 + 2k + 1 = 2(4k^3 + 2k^2 + k) + 1$. Since $4k^3 + 2k^2 + k$ is an integer, by the definition of odd number $n^3$ is odd, but $n^3$ is even. Thus $n$ is even.
    Now suppose that $n$ is even. Then for some integer $k$, $n = 2k$, raising $n$ to the 3rd power we see that $n^3 = (2k)^3 = 8k^3 = 2(4k^3)$. Since $4k^3$ is an integer, then by the definition of even number, $n^3$ is even.
    %12
    \item
    \begin{enumerate}
        \item
    The proof is using the same integer $k$ for the definition of $m$ and $n$, when they might be different.
        \item 
    Counterexample $m = 6, n = 1$
    \end{enumerate}
    %13
    \item
    Let $x$ be an arbitrary real number. Suppose that $\exists x \in \mathbb{R}(x + y = xy)$. Let $y_0$ be some real number such that $x + y_0 = xy_0$. Suppose that $x = 1$ , then $1 + y_0 = 1 \cdot y_0 = y_0$. So $1 = 0$, which is a contradiction, this $x \neq 1$.
    Now suppose that $x \neq 1$. Let $y = \frac{x}{x - 1}$. Then $x + y = x \frac{x}{x - 1} = \frac{x^2}{x -1} = x \cdot \frac{x}{x - 1} = xy$
    %14
    \item
    Let $z = 1$, let $x$ be an arbitrary positive real number. Suppose that $\exists y \in \mathbb{R}(y - x = y / x)$. Let $y = y_0$ such that $y_0 - x = y_0 / x$. Suppose that $x = z = 1$, then $y_0 - 1 = y_0 / 1 = y_0$, thus $-1 = 0$, which is a contradiction, thus $x \neq z$. 
    Now suppose that $x \neq z = 1$. Let $y = \frac{x^2}{x - 1}$, then $y - x = \frac{x^2}{x - 1} - x = \frac{x^2 - x^2 + x}{x - 1} = \frac{x}{x - 1} = \frac{1}{x} \cdot \frac{x^2}{x - 1} = y/x$
    %15
    \item
    Let $x$ be an arbitrary element of $\cup \{A \setminus B \mid A \in \mathscr{F}\}$, then choose some $A$ in $\mathscr{F}$, such that $x \in A \setminus B$, which means that $x \in A$ and $x \notin B$. Since $x \in A$ and $x \notin B$, $A \not \subseteq B$, thus $A \not \in \mathscr{P}(B)$. Since $A \in \mathscr{F}$ and $A \not \in \mathscr{P}(B)$, $A \in \mathscr{F} \setminus \mathscr{P}(B)$. And since $x \in A$, $x \in \mathscr{F} \setminus \mathscr{P}(B)$, since $x$ was an arbitrary element of $\cup \{A \setminus B \mid A \in \mathscr{F}\}$, $\cup \{A \setminus B \mid A \in \mathscr{F}\} \subseteq \cup(\mathscr{F} \setminus \mathscr{P}(B))$
    %16
    \item
    Suppose that $\cup \mathscr{F}$ and $\cap \mathscr{G}$ are not disjoint. Then we can choose some $x$ such that $x \in \cup \mathscr{F}$ and $x \in \cap \mathscr{G}$. Then for some $A \in \mathscr{F}$, $x \in A$. Then for some $B \in \mathscr{G}$ such that $A \cap B = \emptyset$, since $x \in A$, $x \notin B$. But since $x \in \cap \mathscr{G}$ means that $x \in B$, we arrive at a contradiction, thus $\cup \mathscr{F}$ and $\cap \mathscr{G}$ are disjoint.
    %17
    \item
    Let $A$ be an arbitrary set. Suppose $x \in A$. Since $A \subseteq A$, $A \in \mathscr{P}(A)$ , which means that $x \in \cup \mathscr{P}(A)$, since $x$ was an arbitrary element of $A$, $A \subseteq \mathscr{P}(A)$.
    Now supposing that $x \in \cup \mathscr{P}(A)$. Then choosing some $X \in \mathscr{P}(A)$, such that $x \in X$. Since $X \in \mathscr{P}(A)$, $X \subseteq A$, since $x \in X$, $x \in A$. Since $x$ was arbitrary, we can conclude that $\cup \mathscr{P}(A) \subseteq A$.
    Since $A \subseteq \mathscr{P}(A)$ and $\cup \mathscr{P}(A) \subseteq A$, $A = \mathscr{P}(A)$.
    %18
    \item
    \begin{enumerate}
        \item 
    Let $x$ be an arbitrary element of $\cup (\mathscr{F} \cap \mathscr{G})$. Choose some $X$ in $\mathscr{F} \cap \mathscr{G}$ such that $x \in X$. Since $X \in \mathscr{F} \cap \mathscr{G}$, $X \in \mathscr{F}$ and $X \in \mathscr{G}$, thus $X \subseteq \cup \mathscr{F}$ which means that $x \in \cup \mathscr{F}$, similarly we see that $x \in \cup \mathscr{G}$. Thus $x \in (\cup \mathscr{F}) \cap (\cup \mathscr{G})$. Since $x$ was arbitrary $\cup (\mathscr{F} \cap \mathscr{G}) \subseteq (\cup \mathscr{F}) \cap (\cup \mathscr{G})$
        \item 
    The proof uses the same set $A$ for both $\mathscr{F}$ and $\mathscr{G}$ which might not be the case. We cannot assume that both sets are the same.
        \item 
    $\mathscr{F} = \{\{1\}, \{2\}\}$ and $\mathscr{G}\{\{1\}, \{1,2\}\}$
    \end{enumerate}
    %19
    \item
    Suppose that $(\cup \mathscr{F}) \cap (\cup \mathscr{G}) \subseteq \cup (\mathscr{F} \cap \mathscr{G})$. Let $A \in \mathscr{F}$ and $B \in \mathscr{G}$. Suppose that $x \in A \cap B$. Then $x \in A$ and $x \in B$, thus $x \in \cup \mathscr{F}$ and $x \in \cup \mathscr{G}$, which means that $x \in (\cup \mathscr{F}) \cap (\cup \mathscr{G})$, since $x$ was an arbitrary element of $A \cap B$, then $A \cap B \subseteq \cup (\mathscr{F} \cap \mathscr{G})$. 
    
    Now suppose that $\forall A \in \mathscr{F} \forall B \in \mathscr{G}(A \cap B \subseteq \cup (\mathscr{F} \cap \mathscr{G}))$. Let $A$ and $B$ be arbitrary elements of $\mathscr{F}$ and $\mathscr{G}$ respectively. Let $x$ be an arbitrary element of $(\cup \mathscr{F}) \cap (\cup \mathscr{G})$, which means that $x \in \cup \mathscr{F}$ and $x \in \cup \mathscr{G}$, thus $x \in A$ and $x \in B$, since $x \in A \cap B$, and $A \cap B \subseteq \cup (\mathscr{F} \cap \mathscr{G})$, $x \in \cup (\mathscr{F} \cap \mathscr{G})$, since $x$ was an arbitrary element of $(\cup \mathscr{F}) \cap (\cup \mathscr{G})$, $(\cup \mathscr{F}) \cap (\cup \mathscr{G}) \subseteq \cup (\mathscr{F} \cap \mathscr{G})$
    %20
    \item
    Supposing that $\cup \mathscr{F}$ and $\cup \mathscr{G}$ are disjoint. Suppose that $\exists A \in \mathscr{F} \exists B \in \mathscr{G} (A \cap B \neq \emptyset)$. Then choosing some $A \in \mathscr{F}$ and $B \in \mathscr{G}$. Let $x \in A \cap B$, which means that $x \in A$ and $x \in B$ which means that $x \in \cup \mathscr{F}$ and $x \in \cup \mathscr{G}$. But $\cup \mathscr{F}$ and $\cup \mathscr{G}$ are disjoint. Since we arrived at a contradiction, $\forall A \in \mathscr{F} \forall B \in \mathscr{G} (A \cap B = \emptyset)$
    Now supposing that $\forall A \in \mathscr{F} \forall B \in \mathscr{G} (A \cap B = \emptyset)$, suppose that $\cup \mathscr{F}$ and $\cup \mathscr{G}$ are not disjoint. Let $x$ be an arbitrary element of $(\cup \mathscr{F}) \cap (\cup \mathscr{G})$, which means that $x \in \cup \mathscr{F}$ and $x \in \cup \mathscr{G}$. So there exists sets $A \in \mathscr{F}$ and $B \in \mathscr{G}$, such that $x \in A$ and $x \in B$ but $A$ and $B$ are disjoint. Thus $\forall A \in \mathscr{F} \forall B \in \mathscr{G} (A \cap B = \emptyset)$
    %21
    \item
    \begin{enumerate}
        \item 
    Let $x \in (\cup \mathscr{F}) \setminus (\cup \mathscr{G})$, thus $x \in \cup \mathscr{F}$ and $x \notin (\cup \mathscr{G})$. Then for some $A \in \mathscr{F}$, $x \in A$. Supposing that $A \in \mathscr{G}$, then since $x \in A$, $x \in \cup \mathscr{G}$, but $x \notin \cup \mathscr{G}$, thus $A \notin \mathscr{G}$, Which means that $A \in \mathscr{F} \setminus \mathscr{G}$, thus $x \in \cup (\mathscr{F} \setminus \mathscr{G})$
        \item 
    The fact that $A$ is not in $\mathscr{G}$, doesn't meant that there is no other set in $\mathscr{G}$ that contains $x$, thus we cannot derive that $x \notin \cup \mathscr{G}$ from $A \notin \mathscr{G}$ and $A \notin \mathscr{G}$
        \item 
    Supposing that $\cup (\mathscr{F} \setminus \mathscr{G}) \subseteq (\cup \mathscr{F}) \cap (\cup \mathscr{G})$. Suppose that $\exists A \in (\mathscr{F} \setminus \mathscr{G}) \exists B \in \mathscr{G}(A \cap B \neq \emptyset)$. Choosing some $A \in \mathscr{F}$ and some $B \in \mathscr{G}$ such that $x \in A \cap B$, then $x \in A$ and $x \in B$. Since $A \in (\mathscr{F} \setminus \mathscr{G})$, $x \in \cup (\mathscr{F} \setminus \mathscr{G})$ and since $\cup (\mathscr{F} \setminus \mathscr{G}) \subseteq (\cup \mathscr{F}) \setminus (\cup \mathscr{G})$, $x \in (\cup \mathscr{F}) \setminus (\cup \mathscr{G})$, thus $x \in \cup \mathscr{F}$ and $x \notin \cup \mathscr{G}$. Since $x \notin \cup \mathscr{G}$, but $B \in \mathscr{G}$ and $x \in B$, $x \in \cup \mathscr{G}$. Since we arribed at a contradiction $\forall A \in (\mathscr{F} \setminus \mathscr{G}) \forall B \in \mathscr{G} (A \cap B = \emptyset)$.
    Now suppose that $\forall A \in (\mathscr{F} \setminus \mathscr{G}) \forall B \in \mathscr{G} (A \cap B = \emptyset)$. Let $A$ and $B$ be arbitrary. Let $x$ be an arbitrary element of $\cup (\mathscr{F} \setminus \mathscr{G})$. Choosing some $A \in \mathscr{F} \setminus \mathscr{G}$ such that $x \in A$. Thus $A \in \mathscr{F}$ and $A \notin \mathscr{G}$. Then since $A \in \mathscr{F}$, $x \in \cup \mathscr{F}$, and since $A \cap B = \emptyset$, $x \notin B$, thus $x \notin \cup \mathscr{G}$, then $x \in (\cup \mathscr{F}) \setminus (\cup \mathscr{G})$. Since $x$ was an arbitrary element of $\cup (\mathscr{F} \setminus \mathscr{G})$, $\cup (\mathscr{F} \setminus \mathscr{G}) \subseteq (\cup \mathscr{F}) \setminus (\cup \mathscr{G})$
        \item 
    $\mathscr{F} = \{\{1\}, \{1,2\}\}; \mathscr{G} = \{\{1\}, \{2\}\}$
    \end{enumerate}
    %22
    \item
    Supposing that $\cup \mathscr{F} \not \subseteq \cup \mathscr{G}$. Then suppose that $\forall A \in \mathscr{F} \exists B \in \mathscr{G} (A \subseteq B)$. Let $A$ be an arbitrary element of $\mathscr{F}$ and for some $B \in \mathscr{G}$ such that $A \subseteq B$. Let $x$ be an arbitrary element of $A$, then since $A \subseteq B$, $x \in A \land x \in B$. Since $x \in A$ and $A \in \mathscr{F}$, $x \in \cup \mathscr{F}$, similarly, $x \in \cup \mathscr{G}$. Since $x$ was arbitrary this means that $\cup \mathscr{F} \subseteq \cup \mathscr{G}$, but $\cup \mathscr{F} \not \subseteq \cup \mathscr{G}$. Since we arrived at a contradiction $\cup \mathscr{F} \not \subseteq \cup \mathscr{G} \rightarrow \exists A \in \mathscr{F} \forall B \in \mathscr{G} (A \not \subseteq B)$
    %23
    \item
    \begin{enumerate}
        \item Direct proofs using existential instantiation on both directions.
        \item 
    Let $x$ be arbitrary. Suppose that $x \in B \setminus (\cup_{i \in I} A_i)$, then $x \in B$ and $x \not in (\cup_{i \in I} A_i)$. Let $i$ be an arbitrary element of $I$, then $x \not \in A_i$. Since $x \in B$, and $x \not \in A_i$, $x \in B \setminus A_i$. Since $i$ was arbitrary $x \in \cap_{i \in I} (B \setminus A_i)$. Thus $B \setminus \cup_{i \in I} A_i \subseteq \cap_{i \in I} (B \setminus A_i)$.
    Now suppose that $x \in \cap_{i \in I} (B \setminus A_i)$. Let $i$ be an arbitrary element of $I$, then $x \in B \setminus A_i$. Thus $x \in B$ and $x \not \in A_i$. Since $i$ was arbitrary $x \not \in \cup_{i \in I} A_i$, and since $x \in B$, $x \in B \setminus (\cup_{i \in I} A_i)$. Thus $B \setminus (\cup_{i \in I} A_i) = \cap_{i \in I} (B \setminus A_i)$
        \item 
    $B \setminus (\cap_{i \in I} A_i) = \cup_{i \in I} (B \setminus A_i)$.\\
    Let $x$ be arbitrary. Suppose that $x \in B \setminus (\cap_{i \in I} A_i)$. Then $x \in B$ and $x \not \in \cap_{i \in I} A_i$. Choosing some $i \in I$ such that $x \not \in A_i$. Then since $x \in B$ and $x \not \in A_i$, $x \in B \setminus A_i$. Thus $x \in \cup_{i \in I} (B \setminus A_i)$. Since $x$ was arbitrary $\cup_{i \in I} (B \setminus A_i)$ \\
    Now suppose that $x \in \cup_{i \in I}(B \setminus A_i)$. Choosing some $i \in I$ such that $x \in B \setminus A_i$. Then $x \in B$ and $x \not \in A_i$. Since $A_i$ was specific, then $x \not \in \cap_{i \in I} A_i$. Now since $x \in B$ and $x \not \in \cap_{i \in I} A_i$, $x \in B \setminus (\cap_{i \in I} A_i)$. Since $x$ was arbitrary $B \setminus (\cap_{i \in I} A_i)$.
    \end{enumerate}
    %24
    \item
    \begin{enumerate}
        \item 
    Let $x$ be arbitrary. Suppose that $x \in \cup_{i \in I}(A_i \setminus B_i)$. Choose some $i \in I$ such that $x \in A_i \setminus B_i$, Then $x \in A_i$ and $x \not \in B_i$.
    Since $x \in A_i$, $x \in \cup_{i \in I} A_i$, similarly, since $x \not \in B_i$, $x \not \in \cap_{i \in I} B_i$. Thus $x \in (\cup_{i \in I} A_i) \setminus (\cap_{i \in I} B_i)$. Since $x$ was arbitrary, then $\cup_{i \in I}(A_i \setminus B_i) \subseteq (\cup_{i \in I} A_i) \setminus (\cap_{i \in I} B_i)$
        \item 
    $I=\{1,2\}; A_1 = B_1 = \{1\}; A_2 = B_2 = \{2\}$
    \end{enumerate}
    %25
    \item
    \begin{enumerate}
        \item 
    Let $x$ be arbitrary. Suppose that $x \in \cup_{i \in I}(A_i \cap B_i)$. Choosing some $i \in I$ such that $x \in A_i \cap B_i$ then $x \in A_i$ and $x \in B_i$. Since $x \in A_i$, $x \in \cup_{i \in I} A_i$, similarly, since $x \in B_i$, $x \in \cup_{i \in I} B_i$. Since $x \in \cup_{i \in I} A_i$ and $x \in \cup_{i \in I} B_i$, $x \in (\cup_{i \in I} A_i) \cap (\cup_{i \in I} B_i)$. Since $x$ was arbitrary, $\cup_{i \in I}(A_i \cap B_i) \subseteq (\cup_{i \in I} A_i) \cap (\cup_{i \in I} B_i)$
    \end{enumerate}
    %26
    \item
    Let $a$ and $b$ be arbitrary integers. Choosing $c = ab$, then $a \mid c$ and $b \mid c$
    %27
    \item
\end{enumerate}
