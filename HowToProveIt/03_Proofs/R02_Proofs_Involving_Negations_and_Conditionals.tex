Usually it's easier to prove a positive statement than a negative statement, so it is often helpful to reexpress a goal of the form $\lnot P$ before proving it.

\textbf{To prove a goal of the form} $\lnot P$:\\
If possible, reexpress the goal in some other form and then use one of the proof strategies for this other goal form.

Sometimes a goal of the form $\lnot P$ cannot be reexpressed as a positive statement. In this case it is usually best to do a \textit{proof by contradiction}. Start by assumming that $P$ is true, and try to use this assumption to prove something that you know is false. Often this is done by proving a statement that contradicts one of the givens. Because you know that the statement you have proven is false, the assumption that $P$ was true must have been incorrect. The only remaining possibility then is that $P$ is false.

\textbf{To prove a goal of the form} $\lnot P$:\\
Assume $P$ is true and try to reach a contradiction. Once you have reached a contradiction, you can conclude that $P$ must be false.

\textbf{To use a given of the form} $\lnot P$:\\
If you're doing a proof by contradiction, try making $P$ your goal. If you can prove $P$, then the proof will be complete, because $P$ contradicts the given $\lnot P$.
If you're not doing a proof by contradiction, then if possible, reexpress this given in some other form.

\textbf{To use a given of the form} $P \rightarrow Q$:\\
If you are also given $P$, or if you can prove that $P$ is true, then you can use this given to conclude that $Q$ is true. Since it is equivalent to $\lnot Q \rightarrow \lnot P$, if you can prove that $Q$ is false, you can use this given to conclude that $P$ is false.

\textbf{Modus Ponens} If you know that both $P$ and $P \rightarrow Q$ are true, you can conclude that $Q$ must also be true.
\[P\]
\[P \rightarrow Q\]
\[\therefore Q\]

\textbf{Modus Tollens} If you know that $P \rightarrow Q$ is true and $Q$ is false, you can conclude that $P$ must also be false.
\[P \rightarrow Q\]
\[\lnot Q\]
\[\therefore \lnot P\]