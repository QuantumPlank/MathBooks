\newcommand{\case}[1]{\\\hspace*{2em}(Case #1)}
\newcommand{\rdir}{\\$(\rightarrow)$}
\newcommand{\ldir}{\\$(\rightarrow)$}
\newcommand{\powerset}[1]{\mathscr{P}(#1)}

\begin{enumerate}
    % 1
    \item
    Let x be arbitrary. Suppose that $x \in A \cap (B \cup C)$. Then $x \in A$ and $x \in B \lor x \in C$.
    \case{1} Suppose that $x \in B$. Since $x \in A \land x \in B$, $x \in (A \cap B) \cup C$.
    \case{2} Suppose that $x \in C$. Since $x \in A \land x \in C$, $x \in (A \cap B) \cup C$.
    
    Since $x$ was arbitrary, $A \cap (B \cup C) \subseteq (A \cap B) \cup C$.
    % 2
    \item
    Let $x$ be arbitrary. Suppose that $x \in (A \cup B) \setminus C$. Then $(x \in A \lor x \in C) \land x \notin C$.
    \case{1} Suppose that $x \in A$. Since $x \in A \land x\notin C$. $x \in A \cup (B \setminus C)$
    \case{2} Suppose that $x \in B$. Since $x \in A \land x \in B$. $x \in A \cup (B \setminus C)$
    
    Since $x$ was arbitrary, $(A \cup B) \setminus C \subseteq A \cup (B \setminus C)$.
    % 3
    \item
    Let $x$ be arbitrary.
    \rdir Suppose that $x \in A \setminus (A \setminus B)$. Then $x \in A \land x \in B$, thus $x \in A \cap B$.
    \ldir Suppose that $x \in A \cap B$. Then $x \in A \land x \in B$, thus $x \in A \setminus (A \setminus B)$.

    Since $x$ was arbitrary, $A \setminus (A \setminus B) = A \cap B$
    % 4
    \item
    Let $x$ be arbitrary.
    \rdir Suppose $x \in A \setminus (B \setminus C)$, this means that $x \in A \land (x \notin B \lor x \in C)$.
    \case{1} Suppose that $x \notin B$. Then $x \in A \land x \notin B$, thus $x \in (A \setminus B) \cup (A \cap C)$.
    \case{2} Suppose that $x \in C$. Then $x \in A \land x \in C$, thus $x \in (A \setminus B) \cup (A \cap C)$.
    \ldir Suppose that $x \in (A \setminus B) \cup (A \cap C)$, this means that $(x \in A \land x \notin B) \lor (x \in A \land x \in C)$
    \case{1} Suppose that $x \in A \land x \notin B$. Then $x \in A \setminus (B \setminus C)$/
    \case{2} Suppose that $x \in A \land x \in C$. Then $x \in A \setminus (B \setminus C)$

    Since $x$ was arbitrary, $A \setminus (B \setminus C) = (A \setminus B) \cup (A \cap C)$
    % 5
    \item
    Let $x$ be arbitrary.
    Suppose $x \in A$.
    \case{1} Suppose that $x \in C$. Then since $x \in A \cap C$, $x \in B \cap C$, thus $x \in B$.
    \case{2} Suppose that $x \notin C$. Then since $x \in A \cup C$, $x \in B \cup C$, thus $x \in B$.

    Since $x$ was arbitrary $A \subseteq B$.
    % 6
    \item
    Let $x$ be arbitrary. Suppose that $A \triangle B \subseteq A$. Suppose that $B \not \subseteq A$. Then choosing some $x \in B$ such that $x \notin A$. Since $x \in B \land x \notin A$, $x \in B \setminus A$, thus $x \in A \triangle B$, since $A \triangle B \subseteq A$, $x \in A$, but $x \notin A$. Since we arrived at a contradiction, $B \subseteq A$.
    % 7
    \item
    Let $x$ be arbitrary
    \rdir Suppose that $A \cup C \subseteq B \cup C$. Suppose that $x \in A \setminus C$, then $x \in A$ and $x \notin C$ which means that $x \in A \cup C$, and since $A \cup C \subseteq B \cup C$, $x \in B \cup C$. Thus, since $x \notin C$ and $x \in B$, $x \in B \setminus C$.
    \ldir Suppose that $A \setminus C \subseteq B \setminus C$. Suppose that $x \in A \cup C$.
    \case{1} Suppose that $x \in C$. Then $x \in B \cup C$.
    \case{2} Suppose that $x \notin C$. Then since $x \in A$ and $x \notin C$, $x \in A \setminus C$, since $A \setminus C \subseteq B \setminus C$, $x \in B \setminus C$, and since $x \notin C$, $x \in B$. Thus $x \in B \cup C$.
    % 8
    \item
    Let $x$ be arbitrary. Suppose that $x \in \powerset{A} \cup \powerset{B}$.
    \case{1} Suppose that $x \in \powerset{A}$. Let $y$ be an arbitrary element of $x$. Then $y \in A$, so $y \in A \cup B$, which means that $x \subseteq A \cup B$, thus $x \in \powerset{A \cup B}$
    \case{2} By a similar argument supposing that $x \in \powerset{B}$, we arrive at $x \in \powerset{A \cup B}$.
    Since $x$ was arbitrary $\powerset{A} \cup \powerset{B} \subseteq \powerset{A \cup B}$ 
    % 9
    \item
    % 10
    \item
    % 11
    \item
    % 12
    \item
    % 13
    \item
    % 14
    \item
    % 15
    \item
    % 16
    \item
    % 17
    \item
    % 18
    \item
    % 19
    \item
    % 20
    \item
    % 21
    \item
    % 22
    \item
    % 23
    \item
    % 24
    \item
    % 25
    \item
    % 26
    \item
    % 27
    \item
    % 28
    \item
    % 29
    \item
    % 30
    \item
    % 31
    \item
    % 32
    \item
    % 33
    \item
\end{enumerate}
