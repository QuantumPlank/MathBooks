Supposing that one of our givens has the form $P \lor Q$. This given tells us that either $P$ or $Q$ is true, but it doesn't tell us which. Thus, there are two possibilities that we must take into account.

One way to do the proof would be to consider these two possibilities in turn. In other words, first assume that $P$ is true and use this assumption to prove our goal. Then assume $Q$ is true and give another proof that the goal is true.

The two possibilities that are considered separately in this type of proof are called \textit{cases}. The given $P \lor Q$ justifies the use of these two cases by guaranteeing that these cases cover all of the possibilities. Mathematicians say in this situation that the cases are \textit{exhaustive}.

\textbf{To use a given of the form} $P \lor Q$:\\
Break your proof into cases. For case 1, assume that $P$ is true and use this assumption to prove the goal. For case 2, assume that $Q$ is true and give another proof of the goal.
If you are also given $\lnot P$, or you can prove that $P$ is false, then you can use this given to conclude that $Q$ is true. Similarly, if you are given $\lnot Q$ or can prove that $Q$ is false, then you can conclude that $P$ is true.

Proof by cases is sometimes also helpful if you are proving a goal of the form $P \lor Q$. If you can prove $P$ in some cases and $Q$ in others, then as long as your cases are exhaustive you can conclude that $P \lor Q$ is true.

\textbf{To prove a goal of the form} $P \lor Q$:\\
Break your proof into cases. In each case, either prove $P$ or $Q$.
If $P$ is true, then clearly the goal $P \lor Q$ is true, so you only need to worry about the case in which $P$ is false. You can complete the proof in this case by proving that $Q$ is true.
